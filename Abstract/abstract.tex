% ************************** Thesis Abstract *****************************
% Use `abstract' as an option in the document class to print only the titlepage and the abstract.
\begin{abstract}
The capability of the Short-Baseline Near Detector (SBND) to search for Heavy Neutral Leptons (HNLs) is assessed in this thesis.
HNLs are proposed to be the right-handed heavy partner to the left-handed Standard Model (SM) neutrino, motivated by mechanisms of neutrino mass generation.  
HNLs can be produced from kaon decays in the Booster Neutrino Beam (BNB) and subsequently decay inside SBND, producing observable signals.
%This thesis focuses on the HNL channel $N\rightarrow\nu\pi^0$ in the mass range of 140-260 MeV, where the neutral pion decays into di-photon showers inside the detector.
This thesis focuses on HNLs decaying into a neutral pion that results in di-photon showers, of which this channel spans over the mass range of 140-260 MeV.
SBND is a 112 ton liquid argon time projection chamber, which offers an exceptional energy, spatial and timing resolution, enabling the identification of the boosted topology and late arrival features of HNLs compared to SM neutrinos.
In preparation for the search, the characterisation of the readout electronics' timing resolution is outlined and the calibration of charge signals addressing  is also discussed .
%Two selections of HNLs are presented, with one having more aggressive background rejection than the other. 
Selections of HNLs are presented, demonstrating a background rejection efficiency $\mathcal{O}$(10$^{-4}$) while maintaining $\sim30\%$ of signals. 
An assessment of the detector performance under the assumption of an improved timing reconstruction is also given. 
A treatment of statistical and systematics uncertainties is outlined, followed by a procedure to set upper limits on the coupling $|U_{\mu4}|^2$ of Majorana HNLs at the 90\% confidence level. 
Three result scenarios are presented, demonstrating the current and potential capability of SBND. 
\end{abstract}
