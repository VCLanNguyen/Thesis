%!TEX root = ../thesis.tex
%*******************************************************************************
%*********************************** First Chapter *****************************
%*******************************************************************************

\chapter{Introduction}  %Title of the First Chapter
%\addcontentsline{toc}{chapter}{Introduction}
%********************************** %First Section  **************************************

This thesis provides the first assessment of the capability of the Short-Baseline Near Detector (SBND) to search for Heavy Neutral Leptons (HNLs).
Chapter \ref{ChapterHNL} begins with the motivation of HNLs that allows for the generation of Standard Model (SM) neutrino mass.
An overview of a minimal HNL model is given, covering the production and decay of HNLs.
The focus is on kinematically allowed channels that can be produced from the Booster Neutrino Beam (BNB) and subsequently decay inside SBND.

A description of the Liquid Argon Time Projection Chamber (LArTPC) is provided in Chapter \ref{Chapter3}, which is the main detection technology of SBND.
The operating principles are presented, identifying key physical processes of the two main observable signals, ionisation electrons and scintillation photons, that underpin the performance of LArTPCs.

An overview of the SBND and the BNB is covered in Chapter \ref{ChapterDetector}.
The chapter begins with the physics program of SBND, followed by the detector design, describing each subsystem that comprises the detector.
The BNB is discussed next, detailing the beam design and presenting the secondary meson fluxes and neutrino fluxes arriving at SBND.

The simulation framework at SBND is outlined in Chapter \ref{ChapterSim}, to produce Monte Carlo (MC) samples representing data.
A description of different generators to simulate SM neutrinos, cosmic muons and HNLs is first provided.
The HNL generator is covered in detail to illustrate the physics behind the late arrival of HNLs compared to SM neutrinos, which the work presented in later chapters relies upon.  
Finally, the simulation of the particle propagation and the detector response is summarised.

Moreover, the reconstruction framework is provided in Chapter \ref{ChapterReco}, covering the reconstruction for each detection subsystem: (1) TPC, (2) Photon Detection System (PDS) and (3) Cosmic Ray Tagger (CRT).
Specifically in the TPC reconstruction workflow, an update to an algorithm separating track-like and shower-like reconstructed objects is detailed.
An overview of some high-level analysis tools, combining complementary signals from all subsystems, is given next. 
                                                                                                                       
Chapter \ref{ChapterDAQ} outlines the timing performance of the Data Acquisition (DAQ) at SBND.
The chapter begins with a description of the White Rabbit timing system set up to maintain timing synchronisation across different DAQ subsystems.
The timing precision of the readout electronics of the CRT and PDS are then assessed, which are the two detection subsystems with timing resolution $\mathcal{O}$(2 ns).

Charge calibration is discussed in Chapter \ref{ChapterCalib} with two specific studies.
The first study is on the measurement of electron lifetime, performed on MC samples of anode-to-cathode crossing cosmic muon tracks that fully traverse the detector volume.
The second study is to assess the impacts of delta ray fluctuations on recombination, also performed on MC samples with varying delta ray thresholds.  
                                                                                                                                            
Selection procedures to identify HNL signals and reject backgrounds from SM neutrinos and cosmic muons are presented in Chapter \ref{ChapterSelect}.
Signal and background definitions are provided, followed by a description of MC samples used to perform the selection.
Cuts for rejecting tracks from cosmic muons and SM neutrinos are detailed, followed by cuts optimised to identify HNL showers.
Results of the selection are summarised next, followed by a study under the assumption of an improved timing reconstruction.
                                                                                                                                                     
The capabilities of SBND to search for HNLs are assessed in Chapter \ref{ChapterResult}.
The sensitivity is chosen to be the upper limits on the coupling $|U_{\mu4}|^2$ of Majorana HNLs at the 90\% confidence level assuming no detected signals, such that the results can be directly compared against existing limits.
Treatments of uncertainties of HNLs, SM neutrinos and cosmic muons are discussed.
The procedure used to set upper limits is detailed next.
Expected limits are presented for three scenarios that can be achieved at SBND.
A discussion of the results is given with suggestions for future iterations of this work.                        

