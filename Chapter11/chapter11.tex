%!TEX root = ../thesis.tex
%*******************************************************************************
%*********************************** First Chapter *****************************
%*******************************************************************************

\chapter{Conclusions}  %Title of the First Chapter
%\addcontentsline{toc}{chapter}{Conclusions}

%********************************** %First Section  **************************************

This thesis presents the first assessment of the physics capabilities of the Short-Baseline Near Detector (SBND) to search for Heavy Neutral Leptons (HNLs) in the mass range of 140-260 MeV.
This right-handed heavy neutrino state is motivated to provide a mechanism to generate mass for the left-handed SM neutrinos.
HNLs can be produced from the Booster Neutrino Beam (BNB) from meson decays and subsequently decay into SM observables inside SBND. 
Due to having relatively large masses, the two exploitable features of HNLs are their boosted topologies and late arrival compared to SM neutrinos.
Particularly, the late arrival of HNLs can be observed as an excess in the tails of the Gaussian-shaped neutrino beam bucket.
The bucket structure can be reconstructed using the arrival time with a resolution $\mathcal{O}$(2 ns) from PMT signals, as part of the Photon Detection System (PDS).

Towards this goal of achieving nanosecond resolution, the timing performance of readout electronics in the Data Acquisition (DAQ) was evaluated.
An overview of the White Rabbit timing system was given, including the description of the SPEC-TDC as a precise timestamping device with a resolution of 700 ps to record important signals, including beam arrivals and triggers.
This additional timing information enables many high precision physics analyses.
For instance, the SPEC-TDC was used to perform the timing characterisation of the readout electronics of the Cosmic Ray Taggers (CRTs), of which their clock resolutions were determined to be $\mathcal{O}$(2 ns).
Moreover, the SPEC-TDC was used to assess the timing synchronisation of the readout electronics of the PDS.
This resulted in a clock scheme and a correction method to synchronise multiple digitisers, which are vitally important since PMT signals are the ingredients for the timing reconstruction.

Two selection workflows were additionally developed to identify di-photon showers resulting from HNLs while rejecting backgrounds from SM neutrinos and cosmic muons.
One selection is more lenient while the other is more stringent in rejecting backgrounds more aggressively.
Both selections result in a background efficiency of the same magnitude $\mathcal{O}(10^{-4})$ while still maintaining the signal efficiency at $\sim$30\%.
If only bins at the edge of the arrival time distribution are considered, equivalent to the so-called \textit{timing cut}, the background efficiency decreases by two orders of magnitude to $\mathcal{O}(10^{-6})$ while the signal efficiency only reduces to 10\%. 
The timing cut demonstrates the importance of the edge bins having an exceptionally high signal-to-background ratio, which is the driving factor for a competitive sensitivity.
This motivates an assessment of the detector performance under the assumption of an improved in timing reconstruction, resulting in an additional arrival time distribution by smearing true variables.

Building upon the selection previously discussed, setting upper limits on the coupling $|U_{\mu4}|^2$ of HNLs at the confidence level of 90\% was performed on three sets of arrival time distributions after selection on MC samples: (1) lenient, (2) stringent and (3) smeared truth.  
It was found that the stringent distribution results in a more competitive limit than the lenient distribution due to having background-free bins, however, the stringent limit suffers large statistical fluctuations.
These two limits were found to probe the phase space already explored by the MicroBooNE \cite{uboone1, uboone2, uboone3}, NA62 \cite{NA62A, NA62B} and E949 \cite{E949} experiments.
However, they are the first benchmark of the current physics capability of SBND in the search for HNLs.
Moreover, it was demonstrated that the edge bins of the distribution are the key bins that drive the upper limits.   
The smeared true distribution led to the most competitive limits out of the three, surpassing existing results from other experiments.  

The first iteration of searching for HNLs at SBND identifies areas for improvements to be taken as the next steps.
The most important focus should be on improving reconstructing timing and showers.
The results of the smeared true distribution motivate the need for a resolution improvement when reconstructing the arrival time distribution.
This can be achieved by developing a more sophisticated timing reconstruction method as well as incorporating timing information of beam and trigger arrivals recorded by the SPEC-TDC as presented in 
Chapter \ref{ChapterDAQ}.
Moreover, as discussed in Chapter \ref{ChapterSelect}, improving the reconstruction of shower topology and energy will help identify the boosted features of showers resulting from HNLs.
These improvements will not only be beneficial towards the HNL search but many other analyses at SBND.
For example, the timing improvement is applicable for analyses examining the Gaussian tails of the beam bucket, either to look for a new signal or to reject cosmic backgrounds.
On the other hand, the shower improvement is useful to any analyses containing showers in the final states.
These goals are certainly within reach in the next three years of operation for SBND.

%outlook
The search for HNLs at SBND will help improve the sensitivity of HNLs in the mass region of tens to hundreds MeV, with an expected time scale of 2024-2027.
In the near future, the landscape of the coupling $|U_{\mu4}|^2$ illustrated in Fig. \ref{fig:sensitivity} will expect minimal changes until the DUNE experiment comes online in the 2030s-2040s \cite{HNLSilvia, HNLPresentFuture}.
For the projected 12 years of operation, the $1.32 \times 10^{22}$ POT collected by DUNE will provide immense statistics to set competitive upper limits extending beyond existing results in this mass range.
In the case where a discovery of HNLs is made, this will lead to the next steps to understand the properties of HNLs.
Particularly, for neutrino beam experiments like SBND and DUNE, being able to detect and identify the final states of HNLS can help determine the possibility of lepton number violation and thus, the 
Dirac and Majorana nature of HNLs.
Their existence can provide solutions to outstanding questions in particle physics, including the mass mechanism of neutrinos, the nature of dark matter and the observed matter-antimatter asymmetry.
