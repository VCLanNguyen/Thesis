%!TEX root = ../thesis.tex
%*******************************************************************************
%*********************************** First Chapter *****************************
%*******************************************************************************

\chapter{Conclusions}  %Title of the First Chapter
%\addcontentsline{toc}{chapter}{Conclusions}

%********************************** %First Section  **************************************

This thesis presents the first assessment of the physics capabilities of the Short-Baseline Near Detector (SBND) to search for Heavy Neutral Leptons (HNLs) in the mass range of 140--260 MeV.
This search for a right-handed heavy neutrino state is motivated to provide a mass mechanism for the left-handed SM neutrinos.
HNLs are hypothesised to be produced from the Booster Neutrino Beam (BNB) from meson decays and subsequently decay into SM observables inside SBND. 
Due to having mass, the two exploitable features of HNLs are their boosted signal topologies and late arrival compared to SM neutrinos.
Particularly, the late arrival of HNLs can be observed as an excess in the tails of the Gaussian-shaped neutrino beam bucket, which can be reconstructed with a resolution $\mathcal{O}$(2 ns) using the PhotoMultiplier Tubes (PMTs) as part of the Photon Detection System (PDS) of SBND.

Towards this goal of achieving nanosecond resolution of SBND, developments in the scope of the timing performance of the Data AcQuisition (DAQ) were presented in Chapter \ref{ChapterDAQ}.
An overview of the White Rabbit timing system was given, including the description of the SPEC-TDC as a timestamping device with a resolution of 700 ps to record important signals like beams and triggers.
This additional timing information enables many high precision physics analyses.
For instance, the SPEC-TDC was used to perform the timing characterisation of the readout electronics of the Cosmic Ray Taggers (CRTs), of which clock resolutions were determined to be $\mathcal{O}$(2 ns).
Moreover, the SPEC-TDC was used to assess the timing synchronisation of the readout electronics of the PDS, which is vitally important to agree $\mathcal{O}$(1 ns) since PMT signals are used for timing reconstruction.
This resulted in a clock scheme proposal and a correction method to synchronise multiple digitisers.

Two selection workflows were additionally developed to identify di-photon showers resulting from HNLs while rejecting backgrounds from SM neutrinos and cosmic muons.
One selection is more lenient while the other is more stringent in rejecting backgrounds more aggressively.
Both selections result in a background rejection efficiency $\mathcal{O}(10^{-4})$ while still maintaining the signal efficiency at $\sim$30\%.
If only bins at the edge of beam bucket distribution are considered, equivalent to the so-called \textit{timing cut}, the background efficiency increases by two orders of magnitude to $\mathcal{O}(10^{-6})$ while the signal efficiency only reduces to 10\%. 
The timing cut demonstrates the importance of the edge bins having an exceptionally high signal-to-background ratio which is the driving factor for a competitive upper limit.
This further motivates a hypothetical question what if the reconstruction of the beam bucket has a better timing resolution and its consequential impact on the upper limits?
To answer this question, a truth study was also carried out with smearing applied as a pseudo-reconstruction. 

Finally, building upon the selection previously discussed, setting upper limits on the coupling $|U_{\mu4}|^2$ of HNLs at a confidence level of 90\% was performed on three sets of beam bucket distributions after selection on MC samples: (1) lenient, (2) stringent and (3) smeared truth.  
It was found that the stringent beam bucket distribution results in a more competitive limit than the lenient distribution due to having background-free bins, however, the stringent limit suffers large statistical fluctuations.
These two limits were found to exclude the phase space already explored by the MicroBooNE, NA62 and E949 experiments.
However, they are the first benchmark of the current physics capability of SBND in the search for HNLs.
Moreover, it was demonstrated that the edge bins of the beam bucket distribution are the key bins that drive the upper limits.   
The smeared truth beam bucket distribution led to the most competitive limits out of the three, surpassing existing results from other experiments.  

The first iteration of this analysis provides some guidelines applicable to future work.
Firstly, detector systematics should be included in error propagation, with parameters that are impactful to physics measurement identified when SBND is operational.
Secondly, due to the aggressiveness of the background rejection rate, the analysis should be performed on larger background MC samples, $\mathcal{O}$(10$^6$ events) compared to the present $\mathcal{
O}$(10$^5$ events), to avoid statistical fluctuations.
Finally, the results of the smeared truth motivate the need for a resolution improvement when reconstructing the beam bucket distribution.
The preparation work towards the nanosecond resolution has already begun with the DAQ readout electronics and the implementation of a high precision timestamping device.  
A more sophisticated timing reconstruction can be developed incorporating the extra timing information.
This improvement will not only benefit the HNL analysis but any other analysis that exploits the Gaussian tails of the beam bucket to either look for a new signal or to reject cosmogenic backgrounds.
