%!TEX root = ../thesis.tex
%*******************************************************************************
%****************************** Third Chapter **********************************
%*******************************************************************************
\chapter{The Short Baseline Near Detector and The Booster Neutrino Beam}

% **************************** Define Graphics Path **************************
\ifpdf
    \graphicspath{{Chapter4/Figs/Raster/}{Chapter4/Figs/PDF/}{Chapter4/Figs/}}
\else
    \graphicspath{{Chapter4/Figs/Vector/}{Chapter4/Figs/}}
\fi

%********************************** %Opening  **************************************

Chapter 4 Opening

\newpage
%********************************** %First Section  **************************************
\section{The Short-Baseline Near Detector Physics Program}

%SBN Program
%0: SBND, as part of the SBN Program

%Neutrino Cross Section

%Neutrino Oscillation

%BSM

%********************************** %First Section  **************************************
\section{The Short-Baseline Near Detector}

The SBND detector is a 112 tons LArTPC located 110 m from the BNB target.
It is 5 m in length, 4 m in height and 4 m in width.
The detector is made of 2 TPCs sharing the same Cathode Plane Assembly (CPA) at the centre, each with a drift length of 2 m.
A complex Photon Detection System (PDS) is located behind each of the Anode Plane Assemblies (APAs) on the dege of the detection.
The PDS also includes a passive component made up of TPB-coated reflective foils installed at the CPA.
The TPC is placed inside a membrane cryostat, of which is surrounded by seven planes of Cosmic Ray Tagger (CRT) to provide a full coverage of cosmic rejection.

\subsection{Time Projection Chamber}

%TODO: Add 3 figures TPC + Wireplanes + CE

%APA
The APA of SBND is made up of 3 wire planes: two induction planes, referred to as U and V, oriented at an angle $\pm 60^{\circ}$ to the vertical collection plane, referred to as Y, shown as green, blue and red in Fig. \ref{}.
Each wire plane consists of 150 $\mu$m diameter copper-beryllium wires with a spacing of 3 mm.
The wires are tensioned to 7 N to prevent sagging when being cooled down at liquid argon temperature at 87 K\cite{}

%Cold Electronic

%CPA

%Field Cage

\subsection{Photon Detection System}

\subsubsection{Photomultiplier Tubes}
is coated on foils placed on the cathode
, which reflect the incident photon back towards the PDS located behind the anode. 
This also shifts the wavelength of the photon which 

%The timescale of the of the singlet state light travel to detection
%See SBND paper

\subsubsection{X-ARAPUCAs}

\subsection{Cosmic Ray Taggers}

\subsection{Data Acquisition}

\subsection{Trigger}

%********************************** %First Section  **************************************
\section{The Booster Neutrino Beam}

%********************************** %First Section  **************************************
\section{Concluding Remarks}
