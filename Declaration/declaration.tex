% ******************************* Thesis Declaration ***************************

\begin{declaration}

This thesis represents the original work of the author except for where specific references are made to the work of others.
The presented work has not been submitted in whole or in part for consideration for any other degree or qualification in this, or any other university. 
Due to the collaborative nature of particle experiments, the thesis relies upon work performed by collaborators from the Short-Baseline Near Detector (SBND) and other experiments.

The overviews of Heavy Neutral Leptons (HNLs) and physics of liquid argon time projection chambers, given in Chapters \ref{ChapterHNL} and \ref{Chapter3} respectively, contain work to which the author did not contribute.
References assign credit for the work and figures presented.

The overview of SBND given in Chapter \ref{ChapterDetector} relies on work performed by the entire SBND collaboration \cite{SBNProgram,sbnd_det}.
Plots, diagrams and photographs not made by the author are labelled with references to the source.
In the scope of detector installation, the author helped with the cabling of the Photon Detection System (PDS) boxes and their installation to the detector alongside B. Carlson and B. Bogart.
Moreover, the author installed the PDS readout electronics, under the guidance of M. Stancari and W. Badgett.

Also in Chapter \ref{ChapterDetector}, the flux prediction employs the Booster Neutrino Beam simulation developed in MiniBooNE \cite{BNBFlux}.
The flux simulation was performed by Z. Pavlovic and the flux reader was developed by M. Del Tutto. 
The fluxes of SBND were updated with the kaon reweighting scheme from SciBooNE, which the author assisted with the flux validation \cite{SciBooNE}.

In the simulation framework of SBND described in Chapter \ref{ChapterSim}, the author contributed to the development of the MeVPrtl generator for simulating beyond standard model particles.
This generator was a joint effort of SBND and ICARUS collaborators, led by G. Putnam.
The author developed and validated the physics of HNL simulation with R. Alvarez-Garrote and L. Pelegrina-Gutierrez.
The implementation of the generator into SBND was performed together with R. Alvarez-Garrote.
Moreover, the author identified and helped implement a fix in the GENIE generator to enable the timing simulation of neutrino interactions, which enabled the HNL analysis using Monte Carlo samples.

In Chapter \ref{ChapterReco}, the charge reconstruction toolkit, Wirecell \cite{wirecell} and Pandora \cite{pandora}, was developed before the author's involvement.
The author updated the track-shower separation algorithm within Pandora, with the help of E. Tyley and D. Brailsford.
The PDS and CRT reconstruction and analysis tools were contributed by SBND collaborators with credits provided in the references.
Most importantly, the author would like to thank F. Nicolas-Arnaldos, R. Alvarez-Garrote, D. Garcia-Gamez and J. I. Crespo-Anadon, who pioneered the timing reconstruction using the SBND PDS, which the HNL analysis relies on.

The study timing study of the DAQ performance provided in Chapter \ref{ChapterDAQ}, relies on work performed by many SBND collaborators.
The hardware and software of the White Rabbit timing system as well as the CRT Sharps was setup before the author involvement.
The author contributed in the calibration and installation of the SPEC-TDC module and the cabling of timing signals under the guidance of M. Stancari, G. A. Lukhanin and W. Badgett.
The timing characterisation of FEB modules was performed with the help of M. Stancari and H. Lay.
The timing characterisation of CAEN digitisers also received inputs from M. Stancari.  

In Chapter \ref{ChapterCalib}, the presented work in the scope of charge calibration was performed under the guidance of M. Mooney and many discussions with G. Putnam and J. Mueller.
A summary of results from the ICARUS collaboration is included, with references assigned credit in the work and figures presented.

The selection of HNLs presented in Chapter \ref{ChapterSelect} contains many elements in the selection shared across SBND.
MC samples used in the selection were a collaborative work with H. Lay and R. Alvarez-Garrote.
The selection software was built with help from H. Lay. 
Moreover, the cosmic rejection tool CRUMBS was developed by H. Lay, the flash matching OpT0 was developed by L. Tung and the particle identification tool Razzled was developed by H. Lay with the groundwork done by E. Tyley.
Finally, the selection sparked many useful discussions and inputs from R. Alvarez-Garrote, L. Pelegrina-Gutierrez and J. I. Crespo-Anadon.

The uncertainty reweighting in Chapter \ref{ChapterResult} was performed using the framework shared across the SBND and ICARUS collaboration, developed before the author's involvement.
The author would like to thank H. Lay, J. Mueller, and J. Kim for their help in understanding uncertainty treatments.
The procedure to set upper limits on the mixing of HNLs was performed using the \texttt{pyhf} package \cite{pyhf}.
References assign credit for the presented statistical methods employed by \texttt{pyhf} \cite{asymptotic_test, CLs_Junk, CLs_Read}.
%The results of the test statistics were interpreted with advice from T. Junk.
Finally, the author would like to thank A. M. Szelc, whose idea was to search for HNLs by exploiting their lateness, that started this entire thesis.  

% Author and date will be inserted automatically from thesis.tex \author \degreedate

\end{declaration}

